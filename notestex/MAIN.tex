\documentclass[12pt]{article}
\pdfoutput=1 
\usepackage{./notestex/tex/NotesTeX}

\title{
    \begin{center}{\Huge \textit{NotesTeX}}
    \\{{\itshape An All-In-One Notes Package For Students}}\end{center}}
    \author{Aditya Dhumuntarao\footnote{\href{https://geodesick.com/}{\textit{My Personal Website}}}}


\affiliation{
DAMTP, University of Cambridge\\
Perimeter Institute of Theoretical Physics\\
University of Minnesota
}

\emailAdd{adhumunt@gmail.com}

\begin{document}
\maketitle

\flushbottom

\newpage

\pagestyle{fancynotes}

%\part{Introduction}	

\section{Bab 1}\label{sec:sec1}

\begin{margintable}\vspace{.8in}\footnotesize
  \begin{tabularx}{\marginparwidth}{|X}
  Section~\ref{sec:motivation}. Motivation\\
  Section~\ref{sec:reqpackages}. Required Packages\\
  Section~\ref{sec:license}. Margins\\
  \end{tabularx}
\end{margintable}

During my year as a Part III student at Cambridge, I realized that my theoretical physics professors, namely David Tong and David Skinner, would use the \texttt{jhep} format to typeset the notes for their classes. As the year went on, I started typesetting my personal notes during class and realized that the \texttt{jhep} format, while great for publications and lecture notes in general, was lacking a few small but useful features.


\end{document}
