\RequirePackage{luatex85} % tuftebook not yet compatible with recent luatex

\documentclass[twoside,bahasa]{tufte-book}

%\usepackage[T1]{fontenc}
%\usepackage[utf8]{luainputenc}

\setlength{\parskip}{\smallskipamount}
\setlength{\parindent}{0pt}

\usepackage{fontspec}
\setmonofont{DejaVu Sans Mono}

\usepackage{amsmath}
\usepackage{amssymb}


\usepackage{minted}
\newminted{python}{breaklines,fontsize=\footnotesize}
\newminted{text}{breaklines,fontsize=\footnotesize}
\newcommand{\txtinline}[1]{\mintinline[fontsize=\footnotesize]{text}{#1}}
\newcommand{\pyinline}[1]{\mintinline[fontsize=\footnotesize]{python}{#1}}

% Using background color for minted environment
\usepackage{xcolor}
\usepackage{mdframed}

\definecolor{mintedbg}{rgb}{0.95,0.95,0.95}
\BeforeBeginEnvironment{minted}{
    \begin{mdframed}[backgroundcolor=mintedbg,%
        topline=false,bottomline=false,%
        leftline=false,rightline=false]
}
\AfterEndEnvironment{minted}{\end{mdframed}}


\title{Judul}
\author{\noindent{Fadjar Fathurrahman} \\[3mm]
\noindent{Mariya Al Qibtiya Nasution} \\[3mm]}


\usepackage{babel}


\begin{document}
\maketitle

\chapter{Bab 1}

\section{Sub1}

Contoh paragraf. Contoh paragraf. Contoh paragraf. Contoh paragraf.
Contoh paragraf. Contoh paragraf. Contoh paragraf. Contoh paragraf.
Contoh paragraf.

\begin{pythoncode}
def myfunc1(x,y):
    return (x + y) / (x - y)
\end{pythoncode}

Contoh paragraf. Contoh paragraf. Contoh paragraf. Contoh paragraf.
Contoh paragraf. Contoh paragraf. Contoh paragraf. Contoh paragraf.
Contoh paragraf.

\begin{fullwidth}
Paragraf \emph{full-width}. Paragraf \emph{full-width}. Paragraf \emph{full-width}.
Paragraf \emph{full-width}. Paragraf \emph{full-width}. Paragraf \emph{full-width}.
Paragraf \emph{full-width}. Paragraf \emph{full-width}. Paragraf \emph{full-width}.
Paragraf \emph{full-width}. Paragraf \emph{full-width}. Paragraf \emph{full-width}.
\end{fullwidth}


\begin{fullwidth}
\begin{pythoncode}
a_very_very_very_very_long_expr = 1 + 2 + 3 + 4 + sin(x) + cos(x)
\end{pythoncode}
\end{fullwidth}

\section{Sub2}

Contoh paragraf. Contoh paragraf. Contoh paragraf.~Contoh paragraf.
Contoh paragraf. Contoh paragraf.~Contoh paragraf. Contoh paragraf.
Contoh paragraf.
\footnote{Catatan samping. Ini hanya untuk
intermezzo semata. Tapi bisa menambahkan persamaan juga:
\begin{equation}
\alpha + \beta = \frac{\zeta}{\Omega} \int_{-\inf}^{+\inf} G(x)\,\mathrm{d}x
\end{equation}
}


\section{Sub3}

Contoh paragraf. Contoh paragraf. Contoh paragraf.~Contoh paragraf.
Contoh paragraf. Contoh paragraf.~Contoh paragraf. Contoh paragraf.
Contoh paragraf.

Contoh paragraf. Contoh paragraf. Contoh paragraf.~Contoh paragraf.
Contoh paragraf. Contoh paragraf.~Contoh paragraf. Contoh paragraf.
Contoh paragraf.
Contoh paragraf. Contoh paragraf. Contoh paragraf. Contoh paragraf.
Contoh paragraf. Contoh paragraf. Contoh paragraf. Contoh paragraf.
Contoh paragraf.

Contoh paragraf. Contoh paragraf. Contoh paragraf. Contoh paragraf.
Contoh paragraf. Contoh paragraf. Contoh paragraf. Contoh paragraf.
Contoh paragraf.

Contoh paragraf. Contoh paragraf. Contoh paragraf. Contoh paragraf.
Contoh paragraf. Contoh paragraf. Contoh paragraf. Contoh paragraf.
Contoh paragraf.

Contoh paragraf. Contoh paragraf. Contoh paragraf. Contoh paragraf.
Contoh paragraf. Contoh paragraf. Contoh paragraf. Contoh paragraf.
Contoh paragraf.


\chapter{Bab 2}

Contoh paragraf. Contoh paragraf. Contoh paragraf.~Contoh paragraf.
Contoh paragraf. Contoh paragraf.~Contoh paragraf. Contoh paragraf.
Contoh paragraf.

\section{Sub 1}

Contoh paragraf. Contoh paragraf. Contoh paragraf.~Contoh paragraf.
Contoh paragraf. Contoh paragraf.~Contoh paragraf. Contoh paragraf.
Contoh paragraf.

\subsection{Subsub1}
\end{document}
